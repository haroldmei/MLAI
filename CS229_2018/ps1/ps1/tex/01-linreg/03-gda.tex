\clearpage
\item \subquestionpoints{5}
Recall that in GDA we model the joint distribution of $(x, y)$ by the following
equations:
%
\begin{eqnarray*}
	p(y) &=& \begin{cases}
	\phi & \mbox{if~} y = 1 \\
	1 - \phi & \mbox{if~} y = 0 \end{cases} \\
	p(x | y=0) &=& \frac{1}{(2\pi)^{n/2} |\Sigma|^{1/2}}
		\exp\left(-\frac{1}{2}(x-\mu_{0})^T \Sigma^{-1} (x-\mu_{0})\right) \\
	p(x | y=1) &=& \frac{1}{(2\pi)^{n/2} |\Sigma|^{1/2}}
		\exp\left(-\frac{1}{2}(x-\mu_1)^T \Sigma^{-1} (x-\mu_1) \right),
\end{eqnarray*}
%
where $\phi$, $\mu_0$, $\mu_1$, and $\Sigma$ are the parameters of our model.

Suppose we have already fit $\phi$, $\mu_0$, $\mu_1$, and $\Sigma$, and now
want to predict $y$ given a new point $x$. To show that GDA results in a
classifier that has a linear decision boundary, show the posterior distribution
can be written as
%
\begin{equation*}
	p(y = 1\mid x; \phi, \mu_0, \mu_1, \Sigma)
	= \frac{1}{1 + \exp(-(\theta^T x + \theta_0))},
\end{equation*}
%
where $\theta\in\Re^n$ and $\theta_{0}\in\Re$ are appropriate functions of
$\phi$, $\Sigma$, $\mu_0$, and $\mu_1$.

\ifnum\solutions=1{
  \begin{answer}

Derive the posterior distribution from GDA model:

\begin{eqnarray*}
	p(y = 1\mid x)
	&=& \frac{p(x \mid y = 1) p(y = 1)}{p(x \mid y = 0) p(y = 0) + p(x \mid y = 1) p(y = 1)} \\
	&=& \frac {\frac{1}{(2\pi)^{n/2} |\Sigma|^{1/2}} exp \left( -\frac{1}{2}(x-\mu_1)^T \Sigma^{-1}(x-\mu_1)\right) \phi}
		{\frac{1}{(2\pi)^{n/2} |\Sigma|^{1/2}} \left[ exp \left( -\frac{1}{2}(x-\mu_1)^T \Sigma^{-1}(x-\mu_1)\right) \phi
		+ exp \left( -\frac{1}{2}(x-\mu_0)^T \Sigma^{-1}(x-\mu_0)\right) (1-\phi) \right ]} \\
	&=& \frac { \phi exp \left( -\frac{1}{2}(x-\mu_1)^T \Sigma^{-1}(x-\mu_1)\right)}
		{ \phi exp \left( -\frac{1}{2}(x-\mu_1)^T \Sigma^{-1}(x-\mu_1)\right)
		+ (1-\phi) exp \left( -\frac{1}{2}(x-\mu_0)^T \Sigma^{-1}(x-\mu_0)\right)} \\
	&=& \frac {1} {1 + \frac{1-\phi}{\phi} \exp \left (\frac{1}{2}(x-\mu_1)^T \Sigma^{-1}(x-\mu_1) 
		- \frac{1}{2}(x-\mu_0)^T \Sigma^{-1}(x-\mu_0) \right)} \\
	&=& \frac {1} {1 + exp \left( (\mu_0 - \mu_1)^T \Sigma^{-1} x 
		- \frac{1}{2}(\mu_0^T \Sigma^{-1}\mu_0 - \mu_1^T \Sigma^{-1}\mu_1)
		+ log (\frac{1-\phi}{\phi}) \right)}
\end{eqnarray*}

Set parameters $\theta$ and $\theta_0$ and rewrite the posterior:
\begin{eqnarray*}
	\theta &=& \mu_0 - \mu_1)^T \Sigma^{-1} \\
	\theta_0 &=& \frac{1}{2}(\mu_0^T \Sigma^{-1}\mu_0 - \mu_1^T \Sigma^{-1}\mu_1)	+ log (\frac{1-\phi}{\phi}) \\
	p(y = 1\mid x; \phi, \mu_0, \mu_1, \Sigma)
	&=& \frac{1}{1 + \exp(-(\theta^T x + \theta_0))}
\end{eqnarray*}
Q.E.D.
\end{answer}

}\fi
