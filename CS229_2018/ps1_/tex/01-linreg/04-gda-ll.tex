\clearpage
\item \subquestionpoints{7} For this part of the problem only, you may
  assume $n$ (the dimension of $x$) is 1, so that $\Sigma = [\sigma^2]$ is
  just a real number, and likewise the determinant of $\Sigma$ is given by
  $|\Sigma| = \sigma^2$.  Given the dataset, we claim that the maximum
  likelihood estimates of the parameters are given by
  \begin{eqnarray*}
    \phi &=& \frac{1}{m} \sum_{i=1}^m 1\{y^{(i)} = 1\} \\
\mu_{0} &=& \frac{\sum_{i=1}^m 1\{y^{(i)} = {0}\} x^{(i)}}{\sum_{i=1}^m
1\{y^{(i)} = {0}\}} \\
\mu_1 &=& \frac{\sum_{i=1}^m 1\{y^{(i)} = 1\} x^{(i)}}{\sum_{i=1}^m 1\{y^{(i)}
= 1\}} \\
\Sigma &=& \frac{1}{m} \sum_{i=1}^m (x^{(i)} - \mu_{y^{(i)}}) (x^{(i)} -
\mu_{y^{(i)}})^T
  \end{eqnarray*}
  The log-likelihood of the data is
  \begin{eqnarray*}
\ell(\phi, \mu_{0}, \mu_1, \Sigma) &=& \log \prod_{i=1}^m p(x^{(i)} , y^{(i)};
\phi, \mu_{0}, \mu_1, \Sigma) \\
&=& \log \prod_{i=1}^m p(x^{(i)} | y^{(i)}; \mu_{0}, \mu_1, \Sigma) p(y^{(i)};
\phi).
  \end{eqnarray*}
By maximizing $\ell$ with respect to the four parameters,
prove that the maximum likelihood estimates of $\phi$, $\mu_{0}, \mu_1$, and
$\Sigma$ are indeed as given in the formulas above.  (You may assume that there
is at least one positive and one negative example, so that the denominators in
the definitions of $\mu_{0}$ and $\mu_1$ above are non-zero.)

\ifnum\solutions=1 {
  \begin{answer}

The bernoulli pdf can be written as $p(y) = \phi^y (1-\phi)^{(1-y)}$, hence:
\begin{eqnarray*}
\ell(\phi, \mu_{0}, \mu_1, \Sigma) 
	&=& \sum_{i=1}^m \log \phi^{y^{(i)}} (1-\phi)^{(1-y^{(i)})} \\
	&+& \sum_{i=1,y^{(i)}=1}^m \left( -\frac{1}{2}(x^{(i)}-\mu_1)^T \Sigma^{-1} (x^{(i)}-\mu_1) + \log {\frac{1}{(2\pi)^{n/2} |\Sigma|^{1/2}} }\right )\\
	&+& \sum_{i=1,y^{(i)}=0}^m \left( -\frac{1}{2}(x^{(i)}-\mu_0)^T \Sigma^{-1} (x^{(i)}-\mu_0) + \log {\frac{1}{(2\pi)^{n/2} |\Sigma|^{1/2}} }\right )\\
	&=& \sum_{i=1}^m y^{(i)} \log \phi + \sum_{i=1}^m (1-y^{(i)}) \log (1-\phi) \\
\end{eqnarray*}

Gradient w.r.t. parameter $\phi$:
\begin{eqnarray*}
\frac{\partial \ell}{\partial \phi}
	&=& \frac{\partial}{\partial \phi} \left (\sum_{i=1}^m y^{(i)} \log \phi + \sum_{i=1}^m (1-y^{(i)}) \log (1-\phi) \right)\\
	&=& \frac{\sum_{i=1}^m y^{(i)}}{\log \phi} - \frac{\sum_{i=1}^m (1-y^{(i)})}{\log (1-\phi)} \\
\end{eqnarray*}
Set it to 0 vector:
\begin{eqnarray*}
	\frac{\sum_{i=1}^m y^{(i)}}{\log \phi} = \frac{m- \sum_{i=1}^m y^{(i)}} {\log (1-\phi)} \\
	\Rightarrow \phi = \frac{1}{m} \sum_{i=1}^m 1\{y^{(i)} = 1\}
\end{eqnarray*}

Gradient w.r.t. parameter $\mu_1$, the same to $\mu_0$:
\begin{eqnarray*}
\frac{\partial \ell}{\partial \mu_1}
	&=& \frac{\partial}{\partial \mu_1} \sum_{i=1,y^{(i)}=1}^m \left( -\frac{1}{2}(x^{(i)}-\mu_1)^T \Sigma^{-1} (x^{(i)}-\mu_1) 
		+ \log {\frac{1}{(2\pi)^{n/2} |\Sigma|^{1/2}} }\right ) \\
	&=& -\sum_{i=1}^m 1\{y^{(i)}=1\}(x^{(i)}-\mu_1)^T \Sigma^{-1} \\
\end{eqnarray*}
The derivative w.r.t. $\mu_0$ is the same. Set these to 0 vector and get:
\begin{eqnarray*}
	\mu_0 = \frac{\sum_{i=1}^m 1\{y^{(i)} = 0\} x^{(i)}}{\sum_{i=1}^m 1\{y^{(i)} = 0\}} \\
	\mu_1 = \frac{\sum_{i=1}^m 1\{y^{(i)} = 1\} x^{(i)}}{\sum_{i=1}^m 1\{y^{(i)} = 1\}}
\end{eqnarray*}

For the gradient w.r.t. the covariance matrix $\Sigma$, it is still a nxn matrix; 
it has the similar steps of math work. Here we can use $\mu_{y^{(i)}}$ for both 0 and 1 classes. 
\begin{eqnarray*}
\frac{\partial \ell}{\partial \Sigma}
	&=& \frac{\partial}{\partial \Sigma} \sum_{i=1}^m \left( -\frac{1}{2}(x^{(i)}-\mu_{y^{(i)}})^T \Sigma^{-1} (x^{(i)}-\mu_{y^{(i)}}) 
		+ \log {\frac{1}{(2\pi)^{n/2} |\Sigma|^{1/2}} }\right ) \\
	&=& -\frac{\partial}{\partial \Sigma} \sum_{i=1}^m \left( \frac{1}{2}(x^{(i)}-\mu_{y^{(i)}})^T \Sigma^{-1} (x^{(i)}-\mu_{y^{(i)}}) 
		+ \frac{1}{2} \log {|\Sigma|}\right ) \\
	&=& \frac{1}{2} \left (\sum_{i=1}^m (x^{(i)}-\mu_{y^{(i)}})(x^{(i)}-\mu_{y^{(i)}})^T\Sigma^{-2} - m \Sigma^{-1} \right )
\end{eqnarray*}

Again set it to 0 matrix and right multiply by $\Sigma^2$, get $\Sigma$:
\begin{eqnarray*}
\Sigma &=& \frac{1}{m} \sum_{i=1}^m (x^{(i)} - \mu_{y^{(i)}}) (x^{(i)} - \mu_{y^{(i)}})^T
\end{eqnarray*}

\end{answer}

} \fi
