\begin{answer}

PROOF: The model can be written as:
\begin{eqnarray*}
    t &\sim& Bernoulli(\phi_t) \\
    y &\sim& Bernoulli(\phi_y)
\end{eqnarray*}

Given input $x^{(i)}$, the conditional marginal probability of $y^{(i)}$ given $t^{(i)}$ is
\begin{eqnarray*}
p(y^{(i)} = 1\mid x^{(i)}) 
    &=& p(y^{(i)} = 1 \mid t^{(i)} = 1, x^{(i)}) p(t^{(i)} = 1\mid x^{(i)}) \\
    &+& p(y^{(i)} = 1 \mid t^{(i)} = 0, x^{(i)}) p(t^{(i)} = 0\mid x^{(i)}) \\
    &=& p(y^{(i)} = 1 \mid t^{(i)} = 1) p(t^{(i)} = 1\mid x^{(i)}) \\
    &\Rightarrow& \frac{p(y^{(i)} = 1\mid x^{(i)}) } {p(t^{(i)} = 1\mid x^{(i)})} = p(y^{(i)} = 1 \mid t^{(i)} = 1)
\end{eqnarray*}
The last equality uses the conditional independence and the fact that when $t=0$, the probability of $y=1$ is 0. The scale constant
is given by:
\begin{eqnarray*}
\alpha 
    &=& p(y^{(i)} = 1 \mid t^{(i)} = 1) \\
    &=& \frac{p(t^{(i)} = 1 \mid y^{(i)} = 1) p(y^{(i)} = 1)}{p(t^{(i)} = 1)} \\
    &=& \frac{p(y^{(i)} = 1)}{p(t^{(i)} = 1)} \\
    &=& \frac{\phi_y}{\phi_t} 
\end{eqnarray*}

\end{answer}
